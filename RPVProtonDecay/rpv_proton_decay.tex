\documentclass[20pt]{beamer}

\usepackage[utf8x]{inputenc}
\usepackage{default}

\setbeamertemplate{navigation symbols}{}

% ------------------------------------------------------------------------------
\def\TopRow{+1.5}
\def\BottomRow{-1.5}
\def\MiddleRow{0.0}
\def\MiddleRowUp{+0.50}
\def\MiddleRowDown{-0.75}

\def\LeftCol{-2.0}
\def\RightCol{+2.0}
\def\MiddleColOne{-1.0}
\def\MiddleColTwo{+1.0}

\def\ProtonX{-2.5}
\def\ProtonY{0.0}

\def\PionX{+2.5}
% \def\PionY{1.00}
\def\PionY{0.75}
% ------------------------------------------------------------------------------

\usepackage{tikz}
\usetikzlibrary{calc,positioning} % used to give coordinates
\usetikzlibrary{decorations.pathmorphing} % curved/coiled lines for photon and gluon
\usetikzlibrary{decorations.markings} % arrows, etc.
\usetikzlibrary{decorations.pathreplacing}
\usetikzlibrary{shapes}

\begin{document}
\pgfdeclaredecoration{single line}{initial}{ 
  \state{initial}[width=\pgfdecoratedpathlength-1sp]{\pgfpathmoveto{\pgfpointorigin}} 
  \state{final}{\pgfpathlineto{\pgfpointorigin}} 
} 

\tikzset{
  % propagator styles
  susy_spin0/.style={
    dashed,
    line width=2pt,
    red
  },
  %%
  spin0/.style={
    dashed,
    line width=2pt
  },
  %%
  photon/.style={
    decorate,
    decoration={
      snake,
      amplitude=4pt,
      segment length=12pt,
      post length=0mm
    },
    line width=2pt
  },
  %%
  gluon/.style={
    decorate,
    decoration={
      coil,
      amplitude=5pt,
      segment length=8pt,
      aspect=1
    },
    line width=2pt
  },
  %%
  massvect/.style={
    decorate,
    decoration={
      snake
    },
    line width=2pt
  },
  %%
  fermion/.style={
    line width=2pt,
    postaction={
      decorate,
      decoration={
        markings,
        mark=at position 0.55 with{
          \arrow{
            stealth
          }
        }
      }
    }
  },
  %%
  toextpot/.style={
    postaction={
      decorate,
      decoration={
        markings,
        mark=at position 1 with {
          \draw (-2pt,-2pt) -- (2pt,2pt);\draw (-2pt,2pt) -- (2pt,-2pt);
        }
      }
    }
  },
  %%
  arclabel/.style={
    preaction={
      decorate,
      decoration={
        markings,
        mark=at position .5 with {
          \node[void] at (0, 0) [label=#1]{};
        }
      }
    }
  },
  %%
  % These prevent the path from being drawn -> needs to be done twice
  %%
  momarrowr/.style={
    decorate,
    decoration={
      markings,
      mark=at position .5 with {
        \draw[->] (-2.5mm,
        -2.5mm) -- node [label=#1]{} (2.5mm,-2.5mm);
      }
    }
  },
  %%
  momarrowl/.style={
    decorate,
    decoration={
      markings,
      mark=at position .5 with {
        \draw[->] (-2.5mm,+2.5mm) -- node [label=#1]{} (2.5mm,2.5mm);
      }
    }
  },
  %%
  % node styles (internal vertices, ``blobs'', external line starting points)
  vertex/.style={
    circle,
    draw,
    fill=black,
    inner sep=0pt,
    minimum size=.8mm
  },
  %%
  blob/.style={
    circle,
    draw=black!100,
    fill=black!35,
    inner sep=1pt,
    minimum size=5mm
  },
  %%
  oval/.style={
    ellipse,
    minimum height=13mm,
    minimum width=11mm,
    draw=black!100,
    fill=black!35,
    inner sep=1pt,
    line width=2pt
  },
  %%
  void/.style={
    inner sep=0pt,
    minimum size=0pt
  },
  %%
  counterterm/.style={
    lamp,
    draw,
    inner sep=0pt,
    minimum size=6pt
  },
  %%
  raiseline/.style={
    decorate,
    decoration={
      single line,
      raise=#1
    },
    line width=2pt
  },
  %%
  rpv/.style={
    circle,
    draw=blue,
    fill=blue!35,
    inner sep=1pt,
    minimum size=3mm,
    line width=2pt
  }
  %%
}

% ------------------------------------------------------------------------------
\begin{frame}[plain]
  \begin{center}
    \begin{tikzpicture}[scale=1.3]
      %%
      % - - - - - - - - - - - - - - - - - - - - - - - - - - - - - - - - - - -
      % Input particles
      % - - - - - - - - - - - - - - - - - - - - - - - - - - - - - - - - - - -
      %%
      \node[void]
      % (in_u_1) at (-2, +1.5) [
      (in_u_1) at (\LeftCol, \TopRow) [
        label={
          [label distance=0pt]+180:{
            $u$
          }
        }
      ]{}
      {};
      %%
      \node[void]
      (in_u_2) at (\LeftCol, \BottomRow) [
        label={
          [label distance=0pt]+180:{
            $u$
          }
        }
      ]{}
      {};
      %%
      \node[void]
      (in_d) at (\LeftCol, \MiddleRow) [
        label={
          [label distance=0pt]+180:{
            $d$
          }
        }
      ]{}
      {};
      %%

      % - - - - - - - - - - - - - - - - - - - - - - - - - - - - - - - - - - -
      % Output particles
      % - - - - - - - - - - - - - - - - - - - - - - - - - - - - - - - - - - -
      %%
      \node[void]
      (out_u_1) at (\RightCol, \TopRow) [
        label={
          [label distance=0pt]+0:{
            $u$
          }
        }
      ]{}
      {};
      %%
      \node[void]
      % (out_u_2) at (\RightCol, \MiddleRowUp) [
      (out_u_2) at (\RightCol, \MiddleRow) [
        label={
          [label distance=0pt]+0:{
            $\bar{u}$
          }
        }
      ]{}
      {};
      %%
      \node[void]
      (out_e) at (\RightCol, \BottomRow) [
        label={
          [label distance=0pt]+0:{
            $e^{+}$
          }
        }
      ]{}
      {};
      %%

      % - - - - - - - - - - - - - - - - - - - - - - - - - - - - - - - - - - -
      % Internal nodes
      % - - - - - - - - - - - - - - - - - - - - - - - - - - - - - - - - - - -
      %%
      \node[rpv]
      (vert_1) at (\MiddleColOne, \MiddleRowDown) [
        label={
          [label distance=0pt]+90:{
            $\lambda''$
          }
        }
      ]{}
      {};
      %%
      \node[rpv]
      (vert_2) at (\MiddleColTwo, \MiddleRowDown) [
        label={
          [label distance=0pt]+90:{
            $\lambda'$
          }
        }
      ]{}
      {};



      \draw[fermion] (in_u_1) -- (out_u_1);

      \draw[fermion] (in_u_2) -- (vert_1);
      \draw[fermion] (in_d)   -- (vert_1);

      \draw[fermion] (out_u_2) -- (vert_2);
      \draw[fermion] (out_e)   -- (vert_2);

      % \draw[susy_spin0] (vert_1) -- (vert_2) [label={[label distance=5pt]+90:{$\tilde{t}$}}]{}   (v_u_int_1);
      \draw[susy_spin0] (vert_1) -- node
      [
        label={
          [label distance=0pt]+270:{
            $\tilde{d}$
          }
        }
      ]{}
      (vert_2);

      % \draw[ellipse,
      %   minimum height=13mm,
      %   minimum width=10mm,
      %   draw=black!100,
      %   inner sep=1pt,
      %   line width=2pt
      % ] (in_u_2);

      \node[rectangle,
        rounded corners,
        minimum height=50mm,
        minimum width=10mm,
        draw=black!50!white,
        inner sep=1pt,
        line width=2pt,
      ] (proton) at (\ProtonX, \ProtonY) [
        label={
          [label distance=0pt]+110:{
            $p$
          }
        }
      ]{}
      {};

      \node[rectangle,
        rounded corners,
        % minimum height=20mm,
        minimum height=30mm,
        minimum width=10mm,
        draw=black!50!white,
        inner sep=1pt,
        line width=2pt,
      ] (proton) at (\PionX, \PionY) [
        label={
          [label distance=0pt]+20:{
            $\pi^{0}$
          }
        }
      ]{}
      {};

      % \node[void] (out_u_2) at (3.5,+0.50) [label={[label distance=0pt]+0:{$\ell^{+}$}}]{} {};
      % \node[void] (out_d_2) at (3.5,-0.50) [label={[label distance=0pt]+0:{$\ell^{-}$}}]{} {};

      % % Input proton 1
      % \draw[raiseline=0]  (in_1) -- node [label={[label distance=40pt]+147:{p}}]{} (emptyblob);
      % \draw[raiseline=4]  (in_1) -- (emptyblob);
      % \draw[raiseline=-4] (in_1) -- (emptyblob);
      % 
      % % Input proton 2
      % \draw[raiseline=0]  (in_2) -- node [label={[label distance=40pt]-147:{p}}]{} (emptyblob);
      % \draw[raiseline=4]  (in_2) -- (emptyblob);
      % \draw[raiseline=-4] (in_2) -- (emptyblob);
      % 
      % % stop decay
      % \draw[susy_spin0] (emptyblob) -- node [label={[label distance=5pt]+90:{$\tilde{t}$}}]{}   (v_u_int_1);
      % \draw[fermion]    (v_u_int_1) -- (out_u_1);
      % \draw[fermion]    (out_u_2) -- (v_u_int_1);
      % 
      % % anti-stop decay
      % \draw[susy_spin0] (emptyblob) -- node [label={[label distance=5pt]-90:{$\tilde{t}^{*}$}}]{}         (v_d_int_1);
      % \draw[fermion] (out_d_1) -- (v_d_int_1);
      % \draw[fermion] (v_d_int_1) -- (out_d_2);
      % 
      % \node[oval] (blob) at (0,0)  {};

    \end{tikzpicture}
  \end{center}
\end{frame}

\end{document}
