\documentclass[20pt]{beamer}

\usepackage[utf8x]{inputenc}
\usepackage{default}

\setbeamertemplate{navigation symbols}{}

\usepackage{tikz}
\usetikzlibrary{calc,positioning} % used to give coordinates
\usetikzlibrary{decorations.pathmorphing} % curved/coiled lines for photon and gluon
\usetikzlibrary{decorations.markings} % arrows, etc.
\usetikzlibrary{decorations.pathreplacing}
\usetikzlibrary{shapes}

\begin{document}
  \pgfdeclaredecoration{single line}{initial}{ 
    \state{initial}[width=\pgfdecoratedpathlength-1sp]{\pgfpathmoveto{\pgfpointorigin}} 
    \state{final}{\pgfpathlineto{\pgfpointorigin}} 
  } 

  \tikzset{
      % propagator styles
      susy_spin0/.style={dashed,line width=2pt,red},
      spin0/.style={dashed,line width=2pt},
      photon/.style={decorate,decoration={snake,amplitude=4pt,segment length=12pt,post length=0mm},line width=2pt},
      gluon/.style={decorate,decoration={coil,amplitude=5pt,segment length=8pt,aspect=1},line width=2pt},
      massvect/.style={decorate,decoration={snake},line width=2pt},
      fermion/.style={    line width=2pt,postaction={decorate, decoration={markings, mark=at position 0.55 with{\arrow{stealth}}} }},
      toextpot/.style={postaction={decorate,decoration={markings,mark=at position 1 with {\draw (-2pt,-2pt) -- (2pt,2pt);\draw (-2pt,2pt) -- (2pt,-2pt);}}}},
      arclabel/.style={preaction={decorate,decoration={markings,mark=at position .5 with {\node[void] at (0,0) [label=#1]{};}}}},
      %
      % These prevent the path from being drawn -> needs to be done twice
      momarrowr/.style={decorate,decoration={markings,mark=at position .5 with { \draw[->] (-2.5mm,-2.5mm) -- node [label=#1]{} (2.5mm,-2.5mm); }}},
      momarrowl/.style={decorate,decoration={markings,mark=at position .5 with { \draw[->] (-2.5mm,+2.5mm) -- node [label=#1]{} (2.5mm,2.5mm); }}},
      %
      % node styles (internal vertices, ``blobs'', external line starting points)
      vertex/.style={circle,draw,fill=black,inner sep=0pt,minimum size=.8mm},
      blob/.style={circle,draw=black!100,fill=black!35,inner sep=1pt,minimum size=5mm},
      oval/.style={ellipse,minimum height=13mm,minimum width=11mm,draw=black!100,fill=black!35,inner sep=1pt,line width=2pt},
      void/.style={inner sep=0pt,minimum size=0pt},
      counterterm/.style={lamp,draw,inner sep=0pt,minimum size=6pt},
      raiseline/.style={decorate, decoration={single line, raise=#1}, line width=2pt} 
  }

  \begin{frame}[plain]
    \begin{center}
      \begin{tikzpicture}[scale=1.3]
        \node[void] (in_1) at (-2,+1.5) {};
        \node[void] (in_2) at (-2,-1.5) {};

        \node[void] (emptyblob) at (0,0) {};

        \node[void] (v_u_int_1) at (2,+1.5) {};
        \node[void] (v_d_int_1) at (2,-1.5) {};

        \node[void] (out_u_1) at (3.5,+2.50) [label={[label distance=0pt]+0:{b}}]{} {};
        \node[void] (out_d_1) at (3.5,-2.50) [label={[label distance=0pt]+0:{$\bar{\text{b}}$}}]{} {};

        \node[void] (out_u_2) at (3.5,+0.50) [label={[label distance=0pt]+0:{$\ell^{+}$}}]{} {};
        \node[void] (out_d_2) at (3.5,-0.50) [label={[label distance=0pt]+0:{$\ell^{-}$}}]{} {};

        % Input proton 1
        \draw[raiseline=0]  (in_1) -- node [label={[label distance=40pt]+147:{p}}]{} (emptyblob);
        \draw[raiseline=4]  (in_1) -- (emptyblob);
        \draw[raiseline=-4] (in_1) -- (emptyblob);

        % Input proton 2
        \draw[raiseline=0]  (in_2) -- node [label={[label distance=40pt]-147:{p}}]{} (emptyblob);
        \draw[raiseline=4]  (in_2) -- (emptyblob);
        \draw[raiseline=-4] (in_2) -- (emptyblob);

        % stop decay
        \draw[susy_spin0] (emptyblob) -- node [label={[label distance=5pt]+90:{$\tilde{t}$}}]{}   (v_u_int_1);
        \draw[fermion]    (v_u_int_1) -- (out_u_1);
        \draw[fermion]    (out_u_2) -- (v_u_int_1);

        % anti-stop decay
        \draw[susy_spin0] (emptyblob) -- node [label={[label distance=5pt]-90:{$\tilde{t}^{*}$}}]{}         (v_d_int_1);
        \draw[fermion] (out_d_1) -- (v_d_int_1);
        \draw[fermion] (v_d_int_1) -- (out_d_2);

        \node[oval] (blob) at (0,0)  {};

      \end{tikzpicture}
    \end{center}
  \end{frame}

\end{document}
